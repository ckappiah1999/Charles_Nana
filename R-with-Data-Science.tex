% Options for packages loaded elsewhere
\PassOptionsToPackage{unicode}{hyperref}
\PassOptionsToPackage{hyphens}{url}
%
\documentclass[
]{article}
\usepackage{amsmath,amssymb}
\usepackage{iftex}
\ifPDFTeX
  \usepackage[T1]{fontenc}
  \usepackage[utf8]{inputenc}
  \usepackage{textcomp} % provide euro and other symbols
\else % if luatex or xetex
  \usepackage{unicode-math} % this also loads fontspec
  \defaultfontfeatures{Scale=MatchLowercase}
  \defaultfontfeatures[\rmfamily]{Ligatures=TeX,Scale=1}
\fi
\usepackage{lmodern}
\ifPDFTeX\else
  % xetex/luatex font selection
\fi
% Use upquote if available, for straight quotes in verbatim environments
\IfFileExists{upquote.sty}{\usepackage{upquote}}{}
\IfFileExists{microtype.sty}{% use microtype if available
  \usepackage[]{microtype}
  \UseMicrotypeSet[protrusion]{basicmath} % disable protrusion for tt fonts
}{}
\makeatletter
\@ifundefined{KOMAClassName}{% if non-KOMA class
  \IfFileExists{parskip.sty}{%
    \usepackage{parskip}
  }{% else
    \setlength{\parindent}{0pt}
    \setlength{\parskip}{6pt plus 2pt minus 1pt}}
}{% if KOMA class
  \KOMAoptions{parskip=half}}
\makeatother
\usepackage{xcolor}
\usepackage[margin=1in]{geometry}
\usepackage{color}
\usepackage{fancyvrb}
\newcommand{\VerbBar}{|}
\newcommand{\VERB}{\Verb[commandchars=\\\{\}]}
\DefineVerbatimEnvironment{Highlighting}{Verbatim}{commandchars=\\\{\}}
% Add ',fontsize=\small' for more characters per line
\usepackage{framed}
\definecolor{shadecolor}{RGB}{248,248,248}
\newenvironment{Shaded}{\begin{snugshade}}{\end{snugshade}}
\newcommand{\AlertTok}[1]{\textcolor[rgb]{0.94,0.16,0.16}{#1}}
\newcommand{\AnnotationTok}[1]{\textcolor[rgb]{0.56,0.35,0.01}{\textbf{\textit{#1}}}}
\newcommand{\AttributeTok}[1]{\textcolor[rgb]{0.13,0.29,0.53}{#1}}
\newcommand{\BaseNTok}[1]{\textcolor[rgb]{0.00,0.00,0.81}{#1}}
\newcommand{\BuiltInTok}[1]{#1}
\newcommand{\CharTok}[1]{\textcolor[rgb]{0.31,0.60,0.02}{#1}}
\newcommand{\CommentTok}[1]{\textcolor[rgb]{0.56,0.35,0.01}{\textit{#1}}}
\newcommand{\CommentVarTok}[1]{\textcolor[rgb]{0.56,0.35,0.01}{\textbf{\textit{#1}}}}
\newcommand{\ConstantTok}[1]{\textcolor[rgb]{0.56,0.35,0.01}{#1}}
\newcommand{\ControlFlowTok}[1]{\textcolor[rgb]{0.13,0.29,0.53}{\textbf{#1}}}
\newcommand{\DataTypeTok}[1]{\textcolor[rgb]{0.13,0.29,0.53}{#1}}
\newcommand{\DecValTok}[1]{\textcolor[rgb]{0.00,0.00,0.81}{#1}}
\newcommand{\DocumentationTok}[1]{\textcolor[rgb]{0.56,0.35,0.01}{\textbf{\textit{#1}}}}
\newcommand{\ErrorTok}[1]{\textcolor[rgb]{0.64,0.00,0.00}{\textbf{#1}}}
\newcommand{\ExtensionTok}[1]{#1}
\newcommand{\FloatTok}[1]{\textcolor[rgb]{0.00,0.00,0.81}{#1}}
\newcommand{\FunctionTok}[1]{\textcolor[rgb]{0.13,0.29,0.53}{\textbf{#1}}}
\newcommand{\ImportTok}[1]{#1}
\newcommand{\InformationTok}[1]{\textcolor[rgb]{0.56,0.35,0.01}{\textbf{\textit{#1}}}}
\newcommand{\KeywordTok}[1]{\textcolor[rgb]{0.13,0.29,0.53}{\textbf{#1}}}
\newcommand{\NormalTok}[1]{#1}
\newcommand{\OperatorTok}[1]{\textcolor[rgb]{0.81,0.36,0.00}{\textbf{#1}}}
\newcommand{\OtherTok}[1]{\textcolor[rgb]{0.56,0.35,0.01}{#1}}
\newcommand{\PreprocessorTok}[1]{\textcolor[rgb]{0.56,0.35,0.01}{\textit{#1}}}
\newcommand{\RegionMarkerTok}[1]{#1}
\newcommand{\SpecialCharTok}[1]{\textcolor[rgb]{0.81,0.36,0.00}{\textbf{#1}}}
\newcommand{\SpecialStringTok}[1]{\textcolor[rgb]{0.31,0.60,0.02}{#1}}
\newcommand{\StringTok}[1]{\textcolor[rgb]{0.31,0.60,0.02}{#1}}
\newcommand{\VariableTok}[1]{\textcolor[rgb]{0.00,0.00,0.00}{#1}}
\newcommand{\VerbatimStringTok}[1]{\textcolor[rgb]{0.31,0.60,0.02}{#1}}
\newcommand{\WarningTok}[1]{\textcolor[rgb]{0.56,0.35,0.01}{\textbf{\textit{#1}}}}
\usepackage{graphicx}
\makeatletter
\def\maxwidth{\ifdim\Gin@nat@width>\linewidth\linewidth\else\Gin@nat@width\fi}
\def\maxheight{\ifdim\Gin@nat@height>\textheight\textheight\else\Gin@nat@height\fi}
\makeatother
% Scale images if necessary, so that they will not overflow the page
% margins by default, and it is still possible to overwrite the defaults
% using explicit options in \includegraphics[width, height, ...]{}
\setkeys{Gin}{width=\maxwidth,height=\maxheight,keepaspectratio}
% Set default figure placement to htbp
\makeatletter
\def\fps@figure{htbp}
\makeatother
\setlength{\emergencystretch}{3em} % prevent overfull lines
\providecommand{\tightlist}{%
  \setlength{\itemsep}{0pt}\setlength{\parskip}{0pt}}
\setcounter{secnumdepth}{-\maxdimen} % remove section numbering
\ifLuaTeX
  \usepackage{selnolig}  % disable illegal ligatures
\fi
\usepackage{bookmark}
\IfFileExists{xurl.sty}{\usepackage{xurl}}{} % add URL line breaks if available
\urlstyle{same}
\hypersetup{
  pdftitle={I do it for fun},
  pdfauthor={Charles Kwame Appiah},
  hidelinks,
  pdfcreator={LaTeX via pandoc}}

\title{I do it for fun}
\author{Charles Kwame Appiah}
\date{2025-03-24}

\begin{document}
\maketitle

x \textless- ``Hello Charles, I hope you're working hard to become who
you are destined to be by God's purpose?'' x class(x) \# checking the
class of variable x

y \textless- pi\^{}2 \# y class(y) \# checking the class of variable y

z \textless- 15L z class(z) \# checking the class of variable z

a \textless- (5 + 2i)\^{}2 a class(a) \# checking the class of variable
a

l \textless- TRUE class(l) \# checking the class of variable l

x \textless- list(age=c(10,21,22), weight=c(30,33,32)) x names(x) \#
Calling the names of the list x

length(x) \# checking the length of the list x

xk \textless- data.frame(age=c(10,21,22), weight=c(30,33,32)) xk

d \textless- c(``Charles Kwame Appiah is my name'') d class(d) \#
checking the class of variable d length(d)

f\textless- c(1,3,4,6,7) f class(f) \# checking the class of variable f

fo\textless- c(1L,3L,4L,6L,7L) fo class(fo) \# checking the class of
variable fo

\section{Initilization}\label{initilization}

x \textless- vector(mode = ``logical'', length = 5) x class(x)

x{[}1:3{]} \textless- TRUE \# indexing the first to third element with
TRUE x

s \textless- c(TRUE, FALSE,TRUE, 1) s as.logical(s) \# Default

q \textless- list(``Hello World'',2015L, TRUE, 32.1) q
class(q{[}{[}2{]}{]}) \# Checking the class of list 2
class(q{[}{[}4{]}{]}) class(q{[}{[}1{]}{]})

mat \textless- c(2,4,5,7) dim(mat) \textless- c(2,2) \# creating a
matrix with 2 rows and 2 columns mat

temp \textless- c(3,4,5,5.6,6,7) mati \textless- matrix(temp, nrow = 2,
ncol = 3,byrow = TRUE) \# creating a matrix with 2 rows and 3 columns
mati

temp \textless- c(3,4,5,5.6,6,7, 8, 9, 10) mato \textless- matrix(temp,
nrow=3, ncol=3,byrow=TRUE) \# creating a matrix with 3 rows and 3
columns mato

\section{Default byrow = FALSE}\label{default-byrow-false}

temp \textless- c(3,4,5,5.6,6,7) mati \textless- matrix(temp, nrow=2,
ncol=3,byrow=FALSE) mati

t1 \textless- c(23, 55) t2 \textless- c(34, 45)

\section{By rows}\label{by-rows}

rbind(t1,t2) \# binding them by their rows

t3 \textless- c(32, 50) t4 \textless- c(43, 54)

\section{By columns}\label{by-columns}

cbind(t3,t4) \# binding them by their columns

factor \textless- c(``Yes'',``No'',``No'',``Yes'') factor \# use to
encode vectors

f \textless- factor(c(``Yes'',``No'',``No'',``Yes''), levels=
c(``Yes'',``No'')) f

x \textless- NA \# Missing number x

is.na(x) \# checking if it is a missing number

u \textless- 0/0 u class(u) \# checking for the class of variable u

\#Dataframe c \textless- c(``Charles'',``Richmond'',``Nicholas'') d
\textless- c(12, 23, 45) s \textless- c(FALSE,TRUE,TRUE)

dfr \textless- data.frame(Username = c,Age = d, Adult = s) dfr \#
creating a dataframe

\section{First Row}\label{first-row}

dfr{[}1,{]} \# accessing row 1 of the dataframe

\section{First Column}\label{first-column}

dfr{[},1{]} \# accessing column 1 {[}Username{]} of the dataframe

\section{Age Column}\label{age-column}

dfr\$Age

\section{Username Column}\label{username-column}

dfr\$Username

\section{Adult Column}\label{adult-column}

dfr\$Adult

\section{Importing Datasets}\label{importing-datasets}

dat \textless- read.csv(``C:/Users/HP/Desktop/Data Science/Machine
learning/Training r.csv'') print(dat, na.rm = TRUE) dat \textless-
head()

plot( dat,c(itching \textasciitilde{} skin\_rash), color=``red'')

data \textless- read.excel(``C:/Users/HP/Downloads/Copy of V1- UN Data
on Refugees (AiCE \_\_ Dataset).xlsx'')

\section{Sequence}\label{sequence}

v \textless- (10:20) v \# start from 10 and end at 20

w \textless- (-5:9) w \# start from -5 and end at 9

qw \textless- seq(2,34,2) qw \# start from 2 and end at 34 with a moving
step of 2

iqw \textless- seq(2,34,length=6) iqw \# start from 2 and end at 34 with
a length of 6

repe \textless- rep(1:4,4) repe \# repeat 1 to 4, 4 times

eq \textless- rep(``Hello Ann'', 5) eq \# repeat Hello Ann 5 times

we \textless- seq(1,15, 2) we \# start from 1 and end at 15 with a step
of 2

we{[}1:5{]} \# slicing from index 1 to 5

class(we) \# checking the class of we

fo \textless- list(``Hello'',``Hi'',``Hey'') fo fo{[}c(1,2){]}
fo{[}c(1,2,3){]} \# for several elements fo{[}{[}2{]}{]} \# for only one
element

class(fo{[}{[}3{]}{]})

wi \textless- list(age=c(12,23,45), height=c(12.3,45.4, 34.5)) wi
class(wi)

woo \textless- data.frame(age=c(12,23,45), height=c(12.3,45.4, 34.5))
woo class(woo)

wi\(age # accessing only the age list
wi\)height \# accessing only the height list

wi{[}{[}``age''{]}{]} \# accessing only the age list
wi{[}{[}`h',exact=FALSE{]}{]} \# partial matching

class(wi\(age) # checking the class of age list
class(wi\)height) \# checking the class of height list

wr \textless- matrix(1:9, nrow = 3, ncol = 3, by = TRUE) wr \# creating
a matrix with 3 rows and 3 columns class(wr)

class(wr{[}1,1{]}) \# checking the class of row 1 column 1

class(wr{[}1,1, drop=FALSE{]}) \# checking the class of row 1 column 1

ch \textless- c(1:9,NA,NA,NA) ch i\textless- is.na(ch) \# Locating
missing numbers i ch{[}!i{]} \# filtering non missing numbers or is not
missing numbers

\section{Vectorization}\label{vectorization}

ew \textless- rnorm(1000) \#ew er \textless- rnorm(1000) \#er cv
\textless- vector(mode=``numeric'', length=1000) \#cv \# Iteration start
\textless- proc.time() for (i in 1:1000)\{ cv{[}i{]} \textless-
ew{[}i{]} + er{[}i{]} \} proc.time()-start

\section{Vectorization}\label{vectorization-1}

start \textless- proc.time() cv \textless- ew + er proc.time()-start

\section{Control Structures}\label{control-structures}

x \textless- 20 if (x \textless{} 0) \{ print(``Negative!'') \}else if
(x \textless{} 10) \{ print(``Positive, less than 10!'') \}else \{
print(``Number greater than 10!'') \}

x \textless- -20 if (x \textless{} 0) \{ print(``Negative!'') \}else if
(x \textless{} 10) \{ print(``Positive, less than 10!'') \}else \{
print(``Number greater than 10!'') \}

x \textless- 6 if (x \textless{} 0) \{ print(``Negative!'') \}else if (x
\textless{} 10) \{ print(``Positive, less than 10!'') \}else \{
print(``Number greater than 10!'') \}

\section{for loop}\label{for-loop}

for (i in 1:100)\{ cat(i) cat('' ``) \# inserting spaces between the
numbers \}

letters \# lower case

LETTERS \# upper case

class(letters)

for (x in letters)\{ cat(x) cat('' ``) \# inserting spaces between the
letters \}

\section{while loop}\label{while-loop}

x \textless- -1 while (x \textless{} 5)\{ print(x) x \textless- x+1 \}

x\textless- 1 repeat\{ print(x) if (x \textgreater{} 7)\{ break \} x
\textless- x+1 \}

for (i in 1:100)\{ \# Over ride the first 20 iterations if (i \textless=
20)\{ next \} \}

\section{Functions}\label{functions}

myPrinter \textless- function(x)\{ for (i in seq(x))\{ print(``Hello,
Charles'') \} \} myPrinter(3)

volume \textless- function(x=3, y=3, z=3)\{ print(x\emph{y}z) \}
volume(y=3,z=5,x=11)

volume()

myPrinter \textless- function(\ldots, mes)\{ print(sum(\ldots))
print(mes) \} myPrinter (3, 5, 11, mes= ``Hi! Richmond'')

ls() \# displaying objects stored in R currently

\section{Iterated Functions}\label{iterated-functions}

\section{lapply}\label{lapply}

str(lapply)

x \textless- list(a=rnorm(10), b=rnorm(20), c=rnorm(30)) lapply(x, mean)
\# checking the mean of x lapply(x, var)\# checking the variance of x
lapply(x, sd) \# checking the standard deviation of x

\section{sapply}\label{sapply}

str(sapply)

xi \textless- list(a=rnorm(10), b=rnorm(20), c=rnorm(30)) sapply(xi,
mean) \# checking the mean of xi sapply(xi, var) \# checking the
variance of xi sapply(xi, sd) \# checking the standard deviation of xi

\section{Split}\label{split}

dat \textless- data.frame(subject = 1:6,age = c(15, 17, 16,20,21,23),
adult = c(FALSE,FALSE,FALSE,TRUE,TRUE,TRUE)) s \textless- split(dat,
dat\$adult) s \# split them according to True and False

sapply(s, function(x)\{ mean(x{[}{[}``age''{]}{]}) \})

sapply(s, function(x)\{ var(x{[}{[}``age''{]}{]}) \})

sapply(s, function(x)\{ sd(x{[}{[}``age''{]}{]}) \})

\section{tapply}\label{tapply}

str(tapply)

x \textless- c(rnorm(10),rnorm(10),rnorm(10),rnorm(10)) f \textless-
gl(4, 10) f tapply(x, f, mean) tapply(x, f, var) tapply(x, f, sd)

\section{Help}\label{help}

?c ?vector ?sapply ?lapply ?tapply

\section{Types, Quality and Data
preprocessing}\label{types-quality-and-data-preprocessing}

wi

\section{finding each column maximum}\label{finding-each-column-maximum}

m \textless- sapply(wi,max) m

\section{finding each column minimum}\label{finding-each-column-minimum}

n \textless- sapply(wi,min) n

\section{Regualization ith range
{[}0,1{]}}\label{regualization-ith-range-01}

wi\(age <- ( (wi\)age - n{[}1{]})/(m{[}1{]} - n{[}1{]}))\emph{(1 - 0) +
0 wi\(height <- ( (wi\)height - n{[}2{]})/(m{[}2{]} - n{[}2{]}))}(1 - 0)

wi

\section{DPLYR AND TIDYR PACKAGES}\label{dplyr-and-tidyr-packages}

\section{install.packages(``dplyr'')}\label{install.packagesdplyr}

\#library(dplyr) data(airquality) class(airquality) airquality
\textless- tibble(airquality) class(airquality) airquality
select(airquality, Ozone, Solar.R, Day) select(airquality,
-(Wind:Month)) \# offsetting Wind and Month from the airquality dataset
filter(airquality, Month \textgreater{} 5, Month \textless{} 9, Day
\textless{} 3) \# values in Month greater than 5 and less than 9, Day
values less than 3 filter(airquality, Day==1 \textbar{} Day == 2) \#
Values of Day = 1 or 2 arrange(airquality,Ozone, desc(Solar.R))
mutate(airquality, Temp.C = round((Temp - 32) * 5/9)) \# creating a new
column for Temp.C \# Removing rows with missing values on the Ozone and
Solar.R features airquality \textless- filter(airquality,
!is.na(Ozone),!is.na(Solar.R)) airquality \# print(airquality, n=143)

\#Grouping by month by\_month \textless- group\_by (airquality, Month)
by\_month

\#Finding the minimum, average and maximum value per Month summarize
(by\_month, min(Ozone), mean(Ozone), max(Ozone))

\#install.packages (``tidyr'') \#library (tidyr)

\#dat \#gather(dat, sex, count, -subject)

\#dat \textless- gather(dat, sex, class, count, -subject) \#dat

\#dat \#separate(dat, sex, class, c(``sex'', ``class''))

\#dat \#dat \textless- gather(dat, lesson, grade, lesson1:lesson4, na.rm
= TRUE) \#dat

\#dat \textless- spread(dat, quarter, grade) \#dat

\#mutate(dat, lesson = extract\_numeric(lesson))

\section{Statistical Summary and
Visualization}\label{statistical-summary-and-visualization}

\section{Mean}\label{mean}

internet\_usage = c(22,0, 7,12,5, 33, 14, 8, 0, 9) internet\_usage
mean(internet\_usage) \# finding the mean of internet\_usage

\begin{Shaded}
\begin{Highlighting}[]
\NormalTok{net\_usage }\OtherTok{=} \FunctionTok{c}\NormalTok{(}\DecValTok{22}\NormalTok{,}\DecValTok{0}\NormalTok{,}\DecValTok{7}\NormalTok{,}\DecValTok{12}\NormalTok{,}\DecValTok{5}\NormalTok{,}\ConstantTok{NA}\NormalTok{,}\DecValTok{33}\NormalTok{,}\DecValTok{14}\NormalTok{,}\DecValTok{8}\NormalTok{,}\ConstantTok{NA}\NormalTok{,}\DecValTok{0}\NormalTok{,}\DecValTok{9}\NormalTok{)}
\NormalTok{net\_usage}
\end{Highlighting}
\end{Shaded}

\begin{verbatim}
##  [1] 22  0  7 12  5 NA 33 14  8 NA  0  9
\end{verbatim}

\begin{Shaded}
\begin{Highlighting}[]
\FunctionTok{mean}\NormalTok{(net\_usage, }\AttributeTok{na.rm =} \ConstantTok{FALSE}\NormalTok{) }\CommentTok{\# finding the mean of net\_usage with missing numbers }
\end{Highlighting}
\end{Shaded}

\begin{verbatim}
## [1] NA
\end{verbatim}

\begin{Shaded}
\begin{Highlighting}[]
\CommentTok{\# Median}
\FunctionTok{median}\NormalTok{(net\_usage, }\AttributeTok{na.rm =} \ConstantTok{TRUE}\NormalTok{)}
\end{Highlighting}
\end{Shaded}

\begin{verbatim}
## [1] 8.5
\end{verbatim}

\begin{Shaded}
\begin{Highlighting}[]
\CommentTok{\# Minimum, Maximum and Range}
\NormalTok{A }\OtherTok{=} \FunctionTok{c}\NormalTok{(}\DecValTok{49}\NormalTok{,}\DecValTok{33}\NormalTok{,}\DecValTok{63}\NormalTok{,}\DecValTok{48}\NormalTok{,}\DecValTok{54}\NormalTok{,}\DecValTok{62}\NormalTok{,}\DecValTok{52}\NormalTok{,}\DecValTok{64}\NormalTok{,}\DecValTok{71}\NormalTok{,}\DecValTok{68}\NormalTok{)}
\FunctionTok{min}\NormalTok{(A)}
\end{Highlighting}
\end{Shaded}

\begin{verbatim}
## [1] 33
\end{verbatim}

\begin{Shaded}
\begin{Highlighting}[]
\FunctionTok{max}\NormalTok{(A)}
\end{Highlighting}
\end{Shaded}

\begin{verbatim}
## [1] 71
\end{verbatim}

\begin{Shaded}
\begin{Highlighting}[]
\FunctionTok{which.min}\NormalTok{(A)}
\end{Highlighting}
\end{Shaded}

\begin{verbatim}
## [1] 2
\end{verbatim}

\begin{Shaded}
\begin{Highlighting}[]
\FunctionTok{which.max}\NormalTok{(A)}
\end{Highlighting}
\end{Shaded}

\begin{verbatim}
## [1] 9
\end{verbatim}

print(max(A) - min(A)) range(A) print(range(A){[}2{]} - range(A){[}1{]})
summary(A)

\section{Percentile Values}\label{percentile-values}

X = c(3,4,5,6,7,8,10,10,11,12,14,14,14,15,16,17,21,25,27,32)
quantile(X,0.80) quantile(X,0.50,type = 7) quantile(X,0.25,type = 7)
quantile(X,0.75,type = 7) median(X) summary(X) sd(X) \#cv(X)

\section{Interquantile Range}\label{interquantile-range}

irq = function(X) (quantile(X,0.75) - quantile(X,0.25)) irq(X)

\section{Variance and Standard
Deviation}\label{variance-and-standard-deviation}

course = c(6, 2, 1, 9, 17, 4, 3, 2, 1, 5, 11 ,4, 3, 1, 2, 2, 5, 4, 3, 6)
1 / course cf = c(course, 0, course) cf

vw \textless- 2 * course + cf + 1 vw

var(course) sum((course - mean(course)) \^{} 2 / (length(course) - 1))

sd(course) sqrt(var(course)) std = function(x) sqrt(var(x)) std(course)
sqrt(course) sum(course) prod(course) sort(course) order(course)
sqrt(-14 + 9i)

\section{Coefficient of Variation}\label{coefficient-of-variation}

cv = function(x) ( sd(x) / mean(x) ) cv(course)

\section{Visulaization of Qualitative
Data}\label{visulaization-of-qualitative-data}

mo = c(``car'',``car'',``bus'',
``metro'',``metro'',``car'',``metro'',``metro'',``foot'',``car'',``foot'',``bus'',``bus'',``metro'',``metro'',``car'',``car'',``car'',``metro'',``car'')
\# dataset of employee's mo

table(mo) prop.table(table(mo)) data.frame(mo)

\section{Bar Charts}\label{bar-charts}

barplot(table(mo)) hist(table(mo))

barplot(prop.table(table(mo))) hist(prop.table(table(mo)))

\section{Pie Chart}\label{pie-chart}

pie(table(mo), col = c (``red'', ``green'', ``blue'', ``black''))

pie(prop.table(table(mo)), col = c(``purple'', ``green'', ``red'',
``blue''))

\section{Contingency Matrix}\label{contingency-matrix}

g = c(rep(``Male'',8), rep(``Female'',12)) g

mg = table(mo,g) mg

gm = data.frame(mo,g) gm

margin.table(mg, 1) margin.table(mg,2) prop.table(mg) prop.table(mg,1)
prop.table(mg,2)

\section{Stacked Bar Charts and Grouped Bar
Charts}\label{stacked-bar-charts-and-grouped-bar-charts}

barplot(mg, col = c(``red'',``blue'', ``pink'',``green''))

barplot(mg,xlim=c(0,2), xlab=``Gender'', length=levels(mo), col =
c(``red'',``blue'', ``pink'',``green''))

barplot(mg,xlim=c(0,2), xlab=``Transportation'', length=levels(g), col =
c(``red'',``blue'', ``pink'',``green''))

barplot(prop.table(mg,1), width=0.25, xlim=c(0,3), ylim=c(0,1),
xlab=``Gender'', legend=levels(mo), beside=T, col = c(``red'',``blue'',
``pink'',``green''))

mg = table(g,mo) barplot(prop.table(mg,2), width=0.25, xlim=c(0,3),
ylim=c(0,1), xlab=``Transportation'', legend=levels(g), beside=T, col=
c(``red'', ``green''))

xo = c(10, 10, 5, 9, 7, 6,8,6,5,8, 10, 7, 7,8, 5, 6,4,7,9,7, 4,8, 10,10,
7,4,9,5,8,9) table(xo) data.frame(xo)

prop.table(table(xo)) ?prop.table

\section{Histograms}\label{histograms}

hist(xo, col = c(``red'',``blue'', ``pink'',``green'')) hist(xo,
nclass=3, col = c(``red'',``blue'', ``pink'',``green''))

wo = c (1950, 2090, 2700, 3350, 4200, 3720, 4400, 2980, 3850, 4550,
3050, 2350, 1850, 2820, 3670, 2950, 3750, 1850, 2420, 3150, 3000, 3470,
3920, 3100, 2400, 2900, 2650, 3450, 3650, 4020, 4450, 3120, 3660, 3070,
3550, 2020, 3500, 2500, 3780, 3940, 3540, 2800, 4450, 1950, 3020, 2800,
3500, 1480, 4495,2850, 3100, 2250,3300, 4100, 3220, 3600,2130, 4020,
4075) hist(wo, col = c(``red'',``blue'', ``pink'',``green'')) hist(wo,
nclass=4, col = c(``red'',``blue'', ``pink'',``green'')) hist(wo,
breaks= seq(from = 1000, to=5000, by=300), col = c(``red'',``blue'',
``pink'',``green'')) hist(wo, probability=T, col = c(``red'',``blue'',
``pink'',``green'')) rug(jitter(wo))

\section{Frequency Polygon}\label{frequency-polygon}

temp = hist(wo, col = c(``red'',``blue'', ``purple'',``green'')) temp

lines(c(min(temp\(breaks), (temp\)mids),max(temp\(breaks)), c(0,temp\)counts,0),type=``l'')

boxplot(wo, horizontal = T, col = c(``red'')) boxplot(wo, vertical = T,
col = c(``red'')) fivenum(wo) summary(wo)

?fivenum

w1 = c(1950, 2090, 2700, 3350, 4200, 3720, 4400, 2980, 3850, 4550, 3050,
2350,1850, 2820, 3670, 2950, 3750, 1850, 2420,3150, 3000, 3470, 3920,
3100, 2400, 2900, 2650, 3450, 3650, 4020) fivenum(w1)

w2 = c(4450, 3120, 3660, 3070, 3550, 2020, 3500, 2500, 3780, 3940, 3340,
2800, 2850, 4450,
1950,3020,2800,3500,1480,4495,2850,3100,2250,3300,4100,3220,3600,2130,4020,4075)
fivenum(w2)

boxplot(w1, w2, names=c(``sample 1'', ``sample 2''), col = c(``purple'',
``yellow'' ))

\section{Classification and
Prediction}\label{classification-and-prediction}

grad\_desc \textless- function(X, y, theta,alpha, lambda, num\_iters)\{
m \textless- length(y) F\_history \textless- c(rep(0, num\_iters))

for (iter in c(1:num\_iters))\{ temp \textless- vector() temp \textless-
theta * (1 - ((alpha\emph{lambda)/m)) - alpha}(1/m) * (t(X) \%\emph{\%
(X \%}\% theta - y)) theta \textless- temp F\_history{[}iter{]}
\textless- computeCost(X, y, theta) \} print(F\_history{[}num\_iters{]})
return(list(``theta'' = theta, ``F\_history'' = F\_history)) \}

grad\_desc(2,3,5,0.1,7.5,2)

computeCost \textless- function(X, y, th)\{ m \textless- length(y)
return(1/(2\emph{m) } sum((X\%*\%th - y)\^{}2)) \}

computeCost(5,6,7)

\section{Clustering}\label{clustering}

iris\_new \textless- iris \#iris\_new iris\_new\(Species <- NULL
kc <- kmeans(iris_new, 3)
table(iris\)Species, kc\$cluster)

plot(iris\_new{[}c(``Sepal.Length'', ``Sepal.Width''){]}, col =
kc\(cluster)
points(kc\)centers{[},c(``Sepal.Length'',``Sepal.Width''){]}, col = 1:3,
pch=8, cex=2)

plot(iris\_new{[}c(``Petal.Length'', ``Petal.Width''){]}, col =
kc\(cluster)
points(kc\)centers{[},c(``Petal.Length'',``Petal.Width''){]}, col=1:3,
pch=8, cex=2)

data(iris) set.seed(500) idx \textless- sample(1:dim(iris){[}1{]}, 40)
iris\_Sample \textless- iris{[}idx,{]} iris\_Sample\(Species <- NULL
hc <- hclust(dist(iris_Sample), method = "single")
plot(hc, hang = -1, labels = iris\)Species{[}idx{]}, xlab =
``Clusters'') rect.hclust(hc, 3) hc \textless-
hclust(dist(iris\_Sample),method = ``complete'') plot(hc, hang = -1,
labels = iris\$Species{[}idx{]}, xlab = ``Clusters'') rect.hclust(hc, 3)

data(iris) set.seed(500) idx \textless- sample(1:dim(iris){[}1{]}, 40)
iris\_Sample \textless- iris{[}idx,{]} iris\_Sample\(Species <- NULL
hc <- hclust(dist(iris_Sample), method = "complete")
plot(hc, hang = -1, labels = iris\)Species{[}idx{]}, xlab =
``Clusters'') rect.hclust(hc, 3) hc \textless-
hclust(dist(iris\_Sample),method = ``complete'') plot(hc, hang = -1,
labels = iris\$Species{[}idx{]}, xlab = ``Clusters'') rect.hclust(hc, 3)

tr = trees tr

plot(tr{[}c(``Volume'',``Height''){]}, col = ``red'')

plot(tr{[}c(``Girth'',``Height''){]}, col = ``blue'' )

plot(tr{[}c(``Volume'',``Girth''){]}, col = ``black'' )

ca = cars ca plot(ca{[}c(``speed'',``dist''){]})

pre = pressure pre plot(pre{[}c(``temperature'',``pressure''){]},
type=``l'')

cab = CO2 cab plot(cab{[}c(``conc'',``uptake''){]})

oran = Orange oran plot(oran{[}c(``Tree'',``circumference''){]})
plot(oran{[}c(``Tree'',``age''){]})
plot(oran{[}c(``age'',``circumference''){]})

\section{MINING OF FREQUENT ITEMSETS AND ASSOCIATION
RULES}\label{mining-of-frequent-itemsets-and-association-rules}

\section{Arules algorithm}\label{arules-algorithm}

\#install.packages(`arules') library(arules)

db \textless- list(c(``A'', ``B'', ``D'', ``E''), c(``B'', ``C'',
``E''), c(``A'', ``B'', ``D'', ``E''), c(``A'', ``B'', ``C'', ``E''),
c(``A'', ``B'', ``C'', ``D'', ``E''), c(``B'', ``C'', ``D''))

frequent \textless- apriori(db, parameter=list(supp=0.5, conf=1,
target=``frequent itemsets''))

inspect(frequent)

cl \textless- apriori(db, parameter=list(supp=0.5, conf=1,
target=``closed'')) inspect(cl)

mx \textless- apriori(db, parameter=list(supp=0.5, conf=1,
target=``maximal'')) inspect(mx)

rules \textless- apriori(db, parameter=list(supp=0.5, conf=1,
target=``rules'')) inspect(rules)

data(Adult) inspect(apriori(Adult, parameter=list(supp=0.75)))

inspect(apriori(Adult, parameter=list(supp=0.75),
appearance=list(rhs=``capital-gain=None'', default=``lhs'')))

test\_that(``Dimensions are positive'', \{ expect\_error(rectangle(width
= -1, height = 4)) expect\_error(rectangle(width = 2, height = -1))
expect\_error(rectangle(width = -1, height = -1))
expect\_error(rectangle(width = 0, height = 4))
expect\_error(rectangle(width = 2, height = 0))
expect\_error(rectangle(width = 0, height = 0)) \})

seed \textless- as.integer(1000 * rnorm(1)) test\_that(paste(``The test
works with seed'', seed), \{ set.seed(seed) \# test code that uses
random numbers \})

seed()

pi cos(90) cos(1)

\end{document}
